\makeatletter %@ symbol can now be accesed as 11

\title{\ncourse}
\ifx \nauthor\undefined %if the author is undefined, define a default value
    \def\nauthor{Georg Eichler}
\else 
\fi  %fi closes the conditional expression

%\author{Based on lectures by \nlecturer \\ \small Notes taken by \nauthor}
\date{\nterm \ \nyear}


\usepackage{alltt}
\usepackage{amsfonts}
\usepackage{amsmath}
\usepackage{amssymb}
\usepackage{amsthm}
%\usepackage{booktabs}
\usepackage{caption}
\usepackage{enumitem}
\usepackage{fancyhdr}
\usepackage{graphicx}
\usepackage{wrapfig} %allows to float text around images
\usepackage{cancel} %for crossing out in formulas
%\usepackage{mathdots}
%\usepackage{mathtools}
%\usepackage{microtype}
%\usepackage{multirow}
%\usepackage{pdflscape}
\usepackage{pgfplots}
%\usepackage{siunitx}
%\usepackage{slashed}
%\usepackage{tabularx}
\usepackage{tikz}
\usepackage{tkz-euclide}
%\usepackage[normalem]{ulem}
%\usepackage[all]{xy}
\usepackage{imakeidx}
\usepackage{subcaption}


%bibliography
\usepackage{biblatex}
\addbibresource{Literature.bib}

\usepackage{hyperref} %hyperref usually has to be the last imported package
\hypersetup{ %settings how the hyperref will look like
    colorlinks=true,
    linkcolor=blue,
    filecolor=magenta,      
    urlcolor=cyan,
    citecolor = blue,
    pdftitle={Overleaf Example},
    pdfpagemode=FullScreen,
    }

\urlstyle{same}

\makeindex[intoc, title=Index]
\indexsetup{othercode={\lhead{\emph{Index}}}}



\pgfplotsset{compat=1.12}

\pagestyle{fancyplain}
\ifx \ncoursehead \undefined
\def\ncoursehead{\ncourse}
\fi

\lhead{\emph{\nouppercase{\leftmark}}}
\ifx \nextra \undefined
  \rhead{
    \ifnum\thepage=1
    \else
      \npart\ \ncoursehead
    \fi}
\else
  \rhead{
    \ifnum\thepage=1
    \else
      \npart\ \ncoursehead \ (\nextra)
    \fi}
\fi


%\usetikzlibrary{arrows.meta}
%\usetikzlibrary{decorations.markings}
%\usetikzlibrary{decorations.pathmorphing}
%\usetikzlibrary{positioning}
%\usetikzlibrary{fadings}
%\usetikzlibrary{intersections}
\usetikzlibrary{cd} %for drawing commutative diagrams
\usetikzlibrary{patterns}
\usetikzlibrary{babel} %solves conflict with tikz-cd



\ifx \ntrim \undefined
\else
  \usepackage{geometry}
  \geometry{
    papersize={379pt, 699pt},
    textwidth=345pt,
    textheight=596pt,
    left=17pt,
    top=54pt,
    right=17pt
  }
\fi


%theorems
\theoremstyle{definition}
\newtheorem*{definition}{Definition}
\newtheorem*{algorithm}{Algorithm}
\newtheorem{theorem}{Theorem}[section] %square brackets mean a reset of numbering
\newtheorem{lemma}[theorem]{Lemma} 
%[theorem] gives lemma etc. a common numbering with the theorem environment
\newtheorem{corollary}[theorem]{Korollar} 
\newtheorem{proposition}[theorem]{Proposition} 
\newtheorem*{remark}{Bemerkung}
\newtheorem*{note}{Beachte}
\newtheorem*{example}{Beispiel}
\newtheorem*{conjecture}{Vermutung}


%new commands
\newcommand{\term}[1]{\emph{#1}\index{#1}}


%important sets
\newcommand{\C}{\mathbb{C}}
\newcommand{\R}{\mathbb{R}}
\newcommand{\Q}{\mathbb{Q}}
\newcommand{\Z}{\mathbb{Z}}
\newcommand{\N}{\mathbb{N}}
\newcommand{\F}{\mathbb{F}}
\newcommand{\E}{\mathbb{E}}
\newcommand{\Sp}{\mathbb{S}}
\newcommand{\K}{\mathbb{K}}
%\newcommand{\P}{\mathbb{P}}

\DeclareMathOperator*{\sign}{sign}